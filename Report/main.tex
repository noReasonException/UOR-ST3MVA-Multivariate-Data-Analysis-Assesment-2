\documentclass[openany]{article}

%Standard Stefanos Packages
\usepackage[utf8]{inputenc}
\usepackage{dirtytalk}
\usepackage{amsmath}
\usepackage{mathtools}  
\mathtoolsset{showonlyrefs} 
\usepackage{graphicx}
\usepackage{mdframed}
\usepackage{lipsum}
\usepackage{cancel}
\usepackage{systeme}
\usepackage{pgfplots}
\usepackage{textcomp}
\usepackage{geometry}
\usetikzlibrary{arrows}
\geometry{a4paper}
\graphicspath{ {./res/} }
\usepackage{float}
\restylefloat{table}
\usepackage{subcaption}
\newcommand{\comment}[1]{%
	\text{\phantom{(#1)}} \tag{#1}
}
\title{\line(1,0){450}\\ ST3MVA - Multivariate Data Analysis \\ \large{Coursework 2}  \\\line(1,0){450} \\University of Reading, 2021-2022}
\usepackage{pgfplots}
%\author{Stefanos Stefanou}
\newmdtheoremenv{note}{Note}
\pgfplotsset{compat=1.17}

%Extra Packages
\usepackage{tikz}
\usetikzlibrary{automata,positioning}

\usepackage{listings}
\usepackage{xcolor}

\definecolor{dkgreen}{rgb}{0,0.6,0}
\definecolor{gray}{rgb}{0.5,0.5,0.5}
\definecolor{mauve}{rgb}{0.58,0,0.82}

\lstdefinestyle{myScalastyle}{
	frame=tb,
	language=scala,
	aboveskip=3mm,
	belowskip=3mm,
	showstringspaces=false,
	columns=flexible,
	basicstyle={\small\ttfamily},
	numbers=none,
	numberstyle=\tiny\color{gray},
	keywordstyle=\color{blue},
	commentstyle=\color{dkgreen},
	stringstyle=\color{mauve},
	frame=single,
	breaklines=true,
	breakatwhitespace=true,
	tabsize=3,
}
\begin{document}
	\maketitle
	\pagebreak
	
	\section*{Question 1}
%	\begin{lstlisting}[language=R]
%library(dplyr)
%
%data <- read.csv("Spectroscopy.csv")
%data$animal<-recode(data$animal,P="Pig",T="Turkey",C="Cattle")
%attach(data)
%
%wavelengths <- data[,3:17]
%pairs(wavelengths,
%pch=20,
%col=recode(
%data$animal,
%Pig="#bc5090",
%Turkey="#ff6361",
%Cattle="#ffa600"))
%	\end{lstlisting}
	%Spectroscopy-pairs
	
	Notes : Too many observations for star glyphs,faces etc
	Pairs plot more appropriate, but may not reveal groups of variables that distinguish groups, i found 4 clear distinctions 
	with this method, that's sufficient for a first EDA.
	A simple pairs plot can give us insights about the variables that can be used to create groupings
	\begin{figure}[H]
		\begin{subfigure}{\textwidth}
			\centering
			\includegraphics[scale=0.5]{res/Spectroscopy-pairs}
		\end{subfigure}
		
		\begin{subfigure}{\textwidth}
			\centering
			\includegraphics[scale=0.1]{res/Color-Map}
		\end{subfigure}
	\end{figure}
	We can clearly see that the following variables can be used to separating species
	\begin{itemize}
		\item WL8 consistently separates Pig from Turkey and Cattle, when plotted against every other variable
		\item WL14 consistently separates Pig from Turkey and Cattle, when plotted against every other variable
		\item WL10 consistently separates Turkey from Pig and Cattle, when plotted against every other variable
		\item WL12 consistently separates Cattle from Pig and Turkey, when plotted against every other variable
		
	\end{itemize}
	\pagebreak
\end{document}